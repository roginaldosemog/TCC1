\section{Contexto}\label{contextualizacao}

O uso da tecnologia entre os estudantes universitários é destacadamente extenso, e tende a aumentar cada vez mais. Aproximadamente 80\% dos discentes das Instituições Federais de Ensino Superior Brasileiras indicam ter alguma ou muita experiência com computadores, enquanto 19,3\% dizem ter alguma noção e 1,3\% nenhuma noção. Em 2014, a IV Pesquisa havia registrado o percentual de 83,5\%. De todo modo, supera o patamar encontrado em 2010, quando o índice alcançou 78,0\% \cite{fonaprace2019}.

Ao mesmo tempo que se vê o avanço da tecnologia, se observa também o aumento na quantidade de estudos sobre a qualidade de vida dos graduandos, afinal, a mudança de rotina, do conteúdo que deve ser absorvido e do círculo de interações sociais, constituem uma nova fase na vida do ingresso em uma instituição de ensino superior. A rotina de estudos na universidade contribui para amplificar os problemas relativos à saúde mental, exigindo dos estudantes posturas flexíveis e resilientes no ambiente acadêmico \cite{fonaprace2019}.

Além disso, com a pandemia de COVID-19, estudos mostram que sete a cada dez universitários brasileiros tiveram um impacto negativo na sua saúde mental, e dentre esses, 87\% identificaram um aumento da ansiedade e do estresse \cite{chegg2021}.

Segundo a ONU, qualidade de vida é definida como a percepção do indivíduo sobre a sua posição na vida, no contexto da cultura e dos sistemas de valores nos quais ele vive, e em relação a seus objetivos, expectativas, padrões e preocupações \cite{whoqol1995}. Portanto, é possível perceber a importância da avaliação e do cuidado que o estudante universitário deve ter em relação à própria qualidade de vida.

Com o propósito de auxiliar pessoas a terem uma avaliação pessoal em relação à própria vida, foi proposta a ferramenta da Roda da Vida \cite{coachingtools}. Em três passos, ela propõe uma análise visual e crítica de como estão certas áreas da vida de quem a utiliza. O primeiro passo consiste em definir áreas relevantes da própria vida. O próximo é avaliar o quanto a pessoa está satisfeita com cada uma das categorias definidas. Assim, a pessoa que está se avaliando, pode finalizar com o último passo, que é traçar um plano de ações ou atividades que possam ser realizadas no intuito de melhorar a qualidade de vida.