\section{Problema}

Existem diversas ferramentas e plataformas que buscam auxiliar a qualidade de vida do usuário. Porém, os softwares gratuitos que têm saúde mental e qualidade de vida como foco dispõem de poucas funcionalidades, enquanto os softwares pagos podem não ser acessíveis para estudantes que possuem baixa renda.

Sabendo disso, a pergunta de pesquisa definida neste trabalho é:

“Como desenvolver uma ferramenta web gratuita que contribua para a melhoria da qualidade de vida dos estudantes acadêmicos tendo como base os princípios e passos da Roda da Vida?”