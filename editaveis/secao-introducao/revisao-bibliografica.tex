\section{Revisão Bibliográfica}

Existem diversos instrumentos desenvolvidos para avaliar qualidade de vida. Os instrumentos mais utilizados para tal são o WHOQOL-100, e também sua versão abreviada, WHOQOL-BREF, ambos sugeridos pela OMS. O WHOQOL-100 é composto por 100 itens, divididos em 24 subdomínios, que são distribuídos entre seis domínios (físico, psicológico, nível de independência, relações sociais, meio ambiente e religiosidade ou espiritualidade) \cite{whoqol1995}.
Já o WHOQOL-BREF possui 26 itens, e os mesmos 24 subdomínios, que desta vez são distribuídos entre quatro domínios (físico, psicológico, relações sociais e meio ambiente) \cite{whoqol1995}. Essas ferramentas são utilizadas para estudos epidemiológicos com o objetivo de avaliar e planejar sistemas de saúde, e são as mais empregadas em pesquisas sociais \cite{gorenstein2015}.

Outro instrumento de medição de qualidade de vida é o Questionário Genérico de Qualidade de Vida SF-36 que foi desenvolvido especificamente para pesquisas em saúde. Ele fornece um perfil funcional da saúde e do bem-estar do indivíduo, bem como um escore global de suas saúdes física e mental. É um questionário abrangente que pode ser utilizado em qualquer idade e doença, assim como em populações saudáveis \cite{gorenstein2015}. Contudo, sua aplicação se torna inviável a este trabalho pela falta de inclusão de categorias relativas a adequação do sono, funcionamento cognitivo e sexual, funcionamento familiar, autoestima, alimentação, recreação, hobbies, comunicação e sintomas e problemas específicos \cite{gorenstein2015}.

Nota-se que o foco dos instrumentos SF-36, WHOQOL-100 e suas variações, é avaliar a qualidade de vida de um público geral para utilização em estudos populacionais. Pela dinâmica proposta neste trabalho, onde o público alvo são os estudantes universitários, utilizaremos outra ferramenta, a Roda da Vida.

