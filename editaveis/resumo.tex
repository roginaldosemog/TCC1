\begin{resumo}
Com o avanço da tecnologia, é possível observar também o aumento no grau de ansiedade e estresse nos estudantes universitários. É de suma importância a avaliação e o cuidado que o estudante universitário deve ter em relação à própria qualidade de vida. Com o propósito de auxiliar pessoas a terem uma avaliação pessoal em relação à própria vida, foi proposta a Roda da Vida. A pessoa que utiliza a ferramenta pode identificar, em três passos, as áreas da vida na qual está mais insatisfeita, podendo planejar ações e atividades para melhorar essa área. O objetivo deste trabalho é desenvolver uma ferramenta web que auxilie o estudante universitário a se avaliar a partir da Roda da Vida e receber recomendações de atividades que causem um impacto positivo em sua qualidade de vida. Para isso, foram empregados princípios da Revisão Sistemática de Literatura para análise de ferramentas semelhantes e um \textit{survey} para definição das atividades propostas.

 \vspace{\onelineskip}

 \noindent
 \textbf{Palavras-chave}: Qualidade de Vida. Roda da Vida. Engenharia de Software. Progressive Web App.
\end{resumo}
